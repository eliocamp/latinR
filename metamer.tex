\documentclass[runningheads,spanish]{llncs}
%
\usepackage{graphicx}
\usepackage[utf8]{inputenc}
\usepackage{natbib}
\bibliographystyle{abbrvnat}

\usepackage[spanish]{babel}


\begin{document}
%
\title{Si te gusta la estadística, bancate los metámeros}
%
 \titlerunning{Título} 
%
\author{
Elio Campitelli\inst{1} }


\institute{Centro de Investigaciones del Mar y la Atmósfera - CONICET}

%
\maketitle
%

 

\keywords{ Mínimo \and  tres \and  palabras clave }


\section{Introducción}

En 1973 Frank Anscombre creó cuatro sets de datos que compartían la
media y el desvío de cada variable y su coeficiente de correlación pero
que lucían muy distintos cuando se los grafica (ref). Desde entonces, el
cuarteto de Anscombe se usa para ilustrar la importancia de visualizar
los datos crudos en vez de confiar en los estadísticos. A pesar de eso,
no existe mucha investigación sobre el fenómeno general de ``datasets
disintos con los mismos estadísticos'' del cual el cuarteto de Anscombe
es sólo un ejemplo. Además usar un conjunto de datos creados hace 50
años para enseñar esto da la impresión de que es un caso único o
extraordinario.

En este artículo propongo el nombre de ``metámeros estadísticos'' en
analogía al concepto de colorimetría y presento el paquete metamer, que
implementa el algoritmo de blabla (ref) para la creación automática de
metámeros.

\section{Fuente de metamerismo estadístico}

El Demonio de Laplace no sabe ni necesita saber estadística. Él tiene
puede conocer la posición y velocidad de cada partícula del universo y
usar ese conocimiento para predecir su evolución. Pero los seres humanos
no podemos analizar más de unos pocos números por vez. Si queremos
entender el universo tenemos que resumir grand cantidad de observaciones
en unos pocos números. Necesitamos saber estadística.

Ese es el objetivo de la gran mayoría de los métodos estadísticos:
conseguir algunos pocos números representativos y comprensibles a partir
de sets de datos demasiado grandes para entenderlos. El proceso
necesariamente elimina información (distintos métodos priorizan
diferentes parte de la información) lo cual implica necesariamente que
los mismos pocos números pueden ser producidos por una infinidad de sets
de datos distintos. Por ejemplo, se necesitan \(N\) momentos
estadisticos para caracterizar unívocamente un set de datos de \(N\)
observaciones\footnote{Técnicamente unívocamente a menos a menos de una permutación.}.
Como colorario, existen infinitos sets de datos de N observaciones que
comparten los mismos \(n < N\) momentos.

Sea una transformación estadística \(E : A \rightarrow B\). Voy a llamar
metámeros de \(E\) a todo elemento de \(a \in A\) tal que \(E(a) = b_0\)
con \(b_0 \in B\).

El Cuarteto de Anscombe debe ser entendido bajo esta idea. Va más allá
de

Si pensamos en una función que
%
% ---- Bibliography ----
%
% BibTeX users should specify bibliography style 'splncs04'.
% References will then be sorted and formatted in the correct style.
%
% \bibliographystyle{splncs04}

% \bibliography{mybibliography}
%

\bibliography{latinRtest.bib}
\end{document}
