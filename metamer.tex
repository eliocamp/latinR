\documentclass[runningheads,español]{llncs}
%
\usepackage{graphicx}
%\usepackage[utf8]{inputenc}
%\usepackage{natbib}

\usepackage[hidelinks]{hyperref}

\begin{document}

%
\title{Si te gusta la estadística, bancate los metámeros}
%
 \titlerunning{Título} 
%
\author{
Elio Campitelli\inst{1} }


\institute{Centro de Investigaciones del Mar y la Atmósfera - CONICET}

%\addbibresource{}
\maketitle
%

 

\keywords{ Mínimo \and  tres \and  palabras clave }


\hypertarget{introduccion}{%
\section{Introducción}\label{introduccion}}

En 1973 Frank Anscombre creó cuatro sets de datos que compartían la
media y el desvío de cada variable y su coeficiente de correlación pero
que lucían muy distintos cuando se los grafica (Anscombe 1973). Desde
entonces, el cuarteto de Anscombe se usa para ilustrar la importancia de
visualizar los datos crudos en vez de confiar en los estadísticos. A
pesar de eso, no existe mucha investigación sobre el fenómeno general de
``datasets disintos con los mismos estadísticos'' del cual el cuarteto
de Anscombe es sólo un ejemplo. Además usar un conjunto de datos creados
hace 50 años para enseñar esto da la impresión de que es un caso único o
extraordinario.

En este artículo propongo el nombre de ``metámeros estadísticos'' en
analogía al concepto de colorimetría y presento el paquete metamer, que
implementa el algoritmo de Matejka and Fitzmaurice (2017) para la
creación automática de metámeros.

\hypertarget{fuente-de-metamerismo-estadistico}{%
\section{Fuente de metamerismo
estadístico}\label{fuente-de-metamerismo-estadistico}}

El Demonio de Laplace no sabe ni necesita saber estadística. Él tiene
puede conocer la posición y velocidad de cada partícula del universo y
usar ese conocimiento para predecir su evolución. Pero los seres humanos
no podemos analizar más de unos pocos números por vez. Si queremos
entender el universo tenemos que resumir grandes cantidad de
observaciones en unos pocos números. Necesitamos saber estadística.

La mayoría de los métodos estadísticos buscan reducir grandes cantidades
de datos en unos pocos números interpretables y representativos. Esto
implica que son funciones continuas (si dos datasets son similares, sus
estadísticos deben ser similares) que pasan de un espacio de alta
dimensión a uno de dimensión menor. No existen funciones con ambas
propiedades que sean biyectivas (Malek et al. 2010), por lo que, para
cualquier método estadístico, los mismos pocos números pueden ser
producidos por una infinidad de sets de datos distintos. Por ejemplo, se
necesitan \(N\) momentos estadisticos para caracterizar unívocamente un
set de datos de \(N\)
observaciones\footnote{Técnicamente unívocamente a menos a menos de una permutación.}.
Como colorario, existen infinitos sets de datos de N observaciones que
comparten los mismos \(n < N\) momentos.

\begin{definition}
Sea una función $E : A \rightarrow B$ y un $a_0 \in A$. El conjunto $M_{a, E}$ de todos los $a \in A$ tal que $E(a) = E(a_0)$ son los metámeros de $a_0$ con respecto a $E$. 
\end{definition}

Es decir, toda función no biyectiva tiene metámeros. El metamerismo es
una consecuencia inevitable de los métodos estadísticos; no es un bug,
es una característica. El Cuarteto de Anscombe es un ejemplo dramático,
pero no debe entenderse como aplicable sólo a los momentos estadísticos.
Tampoco debe concluirse que visualizar los datos sea la única solución,
ya que al proyectar de los datos en un espacio bidimensional, la
visualización también sufre de metamerismo.

\hypertarget{como-crear-metameros}{%
\section{Cómo crear metámeros}\label{como-crear-metameros}}

El paquete metamer implementa el algoritmo de Matejka and Fitzmaurice
(2017) para generar metámeros a partir de un dataset y una
transformación estadística que debe preservarse. Al ser completamente
genérico, permite ilustrar el metamerismo de cualquier transformación.
Por ejemplo,\ldots.

\begin{verbatim}
library(metamer)

energy <- function(u, v) {
  sum(u^2 + v^2)
}
\end{verbatim}

\hypertarget{bibliografia}{%
\section*{Bibliografía}\label{bibliografia}}
\addcontentsline{toc}{section}{Bibliografía}

\hypertarget{refs}{}
\leavevmode\hypertarget{ref-anscombe1973}{}%
Anscombe, F. J. 1973. ``Graphs in Statistical Analysis.'' \emph{The
American Statistician} 27 (1): 17--21.
\url{https://doi.org/10.2307/2682899}.

\leavevmode\hypertarget{ref-malek2010}{}%
Malek, Freshteh, Hamed Daneshpajouh, Hamidreza Daneshpajouh, and
Johannes Hahn. 2010. ``An Interesting Proof of the Nonexistence
Continuous Bijection Between Rn̂ and R2̂ for N != 2.''
\emph{arXiv:1003.1467 {[}Math{]}}, March.
\url{http://arxiv.org/abs/1003.1467}.

\leavevmode\hypertarget{ref-matejka2017}{}%
Matejka, Justin, and George Fitzmaurice. 2017. ``Same Stats, Different
Graphs: Generating Datasets with Varied Appearance and Identical
Statistics Through Simulated Annealing.'' In \emph{Proceedings of the
2017 CHI Conference on Human Factors in Computing Systems - CHI '17},
1290--4. Denver, Colorado, USA: ACM Press.
\url{https://doi.org/10.1145/3025453.3025912}.
%
% ---- Bibliography ----
%
% BibTeX users should specify bibliography style 'splncs04'.
% References will then be sorted and formatted in the correct style.
%
% \bibliographystyle{splncs04}

% \bibliography{mybibliography}
%

%%\bibliography{metamers.bib}
%
\end{document}
